\documentclass[11pt]{article}

\usepackage{fullpage}
\usepackage{amsmath,amssymb}
\usepackage{iftex}
\ifPDFTeX
  \usepackage[T1]{fontenc}
  \usepackage[utf8]{inputenc}
  \usepackage{textcomp} % provide euro and other symbols
\else % if luatex or xetex
  \usepackage{unicode-math} % this also loads fontspec
  \defaultfontfeatures{Scale=MatchLowercase}
  \defaultfontfeatures[\rmfamily]{Ligatures=TeX,Scale=1}
\fi
\usepackage{lmodern}
\ifPDFTeX\else
  % xetex/luatex font selection
\fi
% Use upquote if available, for straight quotes in verbatim environments
\IfFileExists{upquote.sty}{\usepackage{upquote}}{}
\IfFileExists{microtype.sty}{% use microtype if available
  \usepackage[]{microtype}
  \UseMicrotypeSet[protrusion]{basicmath} % disable protrusion for tt fonts
}{}
\makeatletter
\@ifundefined{KOMAClassName}{% if non-KOMA class
  \IfFileExists{parskip.sty}{%
    \usepackage{parskip}
  }{% else
    \setlength{\parindent}{0pt}
    \setlength{\parskip}{6pt plus 2pt minus 1pt}}
}{% if KOMA class
  \KOMAoptions{parskip=half}}
\makeatother
\usepackage{xcolor}
\setlength{\emergencystretch}{3em} % prevent overfull lines
\providecommand{\tightlist}{%
  \setlength{\itemsep}{0pt}\setlength{\parskip}{0pt}}
\setcounter{secnumdepth}{-\maxdimen} % remove section numbering
\ifLuaTeX
  \usepackage{selnolig}  % disable illegal ligatures
\fi
\IfFileExists{bookmark.sty}{\usepackage{bookmark}}{\usepackage{hyperref}}
\IfFileExists{xurl.sty}{\usepackage{xurl}}{} % add URL line breaks if available
\urlstyle{same}
\hypersetup{
  hidelinks,
  pdfcreator={LaTeX via pandoc}}

\begin{document}

{ARMv8 AArch64 Interim Checkpoint Group 5}

\hypertarget{h.z7gbdr1o83yk}{%
\section{\texorpdfstring{{How we have split and coordinated the
workload:}}{How we have split and coordinated the workload:}}\label{h.z7gbdr1o83yk}}

{}

{After reading the specification, we noticed that, if we had four people
all working on the same part, there would be a lot of redundancy. One
person would often have to wait for another to finish their tasks before
they start their own.}{~Instead, we decided to split the group into two
pairs with one working on the emulator and the other working on the
assembler. Since understanding of how the emulator worked was a
prerequisite to building the assembler, we - as a group - decided to
first discuss a broad approach to implementing the emulator. After this,
Virginia and Jack continued bringing this implementation to fruition,
whilst Bella and Aarit started on the assembler. Throughout this
process, we ensured there was a constant line of communication between
the two sub-groups where we discussed any issues and monitored each
other's progress.}

{}

{In order to coordinate all our work and ensure everyone is being held
accountable, we have all come into the university computer labs at least
4 days a week from 10am-6pm. We have all been sitting together and
maintained constant discussion and communication, allowing for effective
pair programming. This was advantageous, as it reduced coding errors and
we could learn from each other about C. Although it felt slower at the
time, we implemented functions faster than if we had coded individually
and then analysed each other's code in our spare time. For our IDE, we
used REPL initially which allowed us to code simultaneously on the same
functions without having to worry about git merging. However, connecting
REPL to git was problematic and it was hard to manage version control so
we swapped to using CLion.}

{}

\hypertarget{h.xzqlrvx87o4t}{%
\section{\texorpdfstring{{How well we believe the group is working and
how it may change for later
tasks:}}{How well we believe the group is working and how it may change for later tasks:}}\label{h.xzqlrvx87o4t}}

{}

{Overall, we think the group has worked very well and we have made a
significant amount of progress with the emulator fully complete, passing
all the tests. In addition, we have fully implemented the assembler and
we are now working on debugging in order to pass all the tests. However,
there are ways we could have improved, e.g. by using a program to manage
which tasks were left to do and we could have used git issues to
highlight bits of code to work on. Moving forward, we realise we will
have to change the structure of our group. Initially, Aarit and Bella
will work on finishing the assembler while Jack and Virginia get a head
start on the coding for the raspberry pi. Then we will need to work as a
four for the remainder of the third part and the extension exercise,
since these are bigger projects which will require us to work as a team
and share what we have learnt from the first two parts, ensuring
everyone understands each others' code.}

{}

\hypertarget{h.y7qyz74o89sv}{%
\section{\texorpdfstring{{How the emulator is structured and code
reuse:}}{How the emulator is structured and code reuse:}}\label{h.y7qyz74o89sv}}

{}

{We have split the emulator into several different files with
categorised helper functions - they read/write the files, execute shifts
and execute the binary functions (such as getting precise parts of the
instruction). We have made a `Machine' structure, which stores the
memory, registers, the current instruction and the }{PSTATE}{~flags.
This single machine entity is then initialised and used as a global
variable across all the execute functions. The execute functions
(DP-Immediate, Branch, etc.) have also been split into separate files,
along with the arithmetic function - in order to avoid code duplication
between the DP instructions. The main emulator file contains and runs
the fetch, decode, execute cycle. Additionally, we have an extra file
that contains tests for various functions - allowing us to test specific
aspects of our code.}

{}

{Since emulator and assembler are fairly distinct, there have been
limited amounts of code we were able to re-use apart from some helper
functions dealing with file handling and shifting. However,
understanding the code for the emulator was invaluable in implementing
the assembler since understanding the inputs that the emulator required
informed what outputs we needed to return in the assembler. We
ultimately found that by having a greater appreciation for how the
emulator worked greatly aided our understanding of the assembler and so
constantly sharing ideas greatly sped up our progress.}

{}

\hypertarget{h.aw2vc64c15c5}{%
\section{\texorpdfstring{{What we may find difficult later and how we
will mitigate
this:}}{What we may find difficult later and how we will mitigate this:}}\label{h.aw2vc64c15c5}}

{}

{There are currently three sections of the project left: debugging the
assembler, the raspberry pi and the extension.}

{}

\hypertarget{h.9evstpd5lucx}{%
\subsubsection{\texorpdfstring{{Assembler}}{Assembler}}\label{h.9evstpd5lucx}}

{Due to the scale of the project as well as the amount of detail within
it, it is possible that we have missed small details or we have
implemented some functions incorrectly. Therefore we are expecting that
it will take time refactoring code during the debugging process and
identifying issues we need to fix. In order to mitigate this, we are
going to try and map out the logic through diagrams so that we can
better visualise where things may go wrong.}

{}

\hypertarget{h.eo3he32ltesq}{%
\subsubsection{\texorpdfstring{{Raspberry
Pi}}{Raspberry Pi}}\label{h.eo3he32ltesq}}

{None of us have much experience with Raspberry Pis so we are going to
have to make sure we extensively research and experiment with the
hardware to ensure we are familiar with how to use it. }

{}

\hypertarget{h.8pkeeyy43oom}{%
\subsubsection{\texorpdfstring{{Extension}}{Extension}}\label{h.8pkeeyy43oom}}

{Finally, due to the freedom we have to choose what extension we want to
implement, we are hoping this will be the most fun part of the project.
However, ensuring we choose an extension that is enjoyable, but also the
right level of difficulty is something we are going to have to balance
carefully. By keeping these considerations in mind and planning
thoroughly, we can hopefully avoid creating problems for ourselves later
in the process.}

\end{document}
