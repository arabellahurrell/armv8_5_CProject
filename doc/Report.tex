\documentclass[11pt]{article}

\usepackage{fullpage}
\usepackage{amsmath,amssymb}
\usepackage{iftex}
\ifPDFTeX
  \usepackage[T1]{fontenc}
  \usepackage[utf8]{inputenc}
  \usepackage{textcomp} % provide euro and other symbols
\else % if luatex or xetex
  \usepackage{unicode-math} % this also loads fontspec
  \defaultfontfeatures{Scale=MatchLowercase}
  \defaultfontfeatures[\rmfamily]{Ligatures=TeX,Scale=1}
\fi
\usepackage{lmodern}
\ifPDFTeX\else
  % xetex/luatex font selection
\fi
% Use upquote if available, for straight quotes in verbatim environments
\IfFileExists{upquote.sty}{\usepackage{upquote}}{}
\IfFileExists{microtype.sty}{% use microtype if available
  \usepackage[]{microtype}
  \UseMicrotypeSet[protrusion]{basicmath} % disable protrusion for tt fonts
}{}
\makeatletter
\@ifundefined{KOMAClassName}{% if non-KOMA class
  \IfFileExists{parskip.sty}{%
    \usepackage{parskip}
  }{% else
    \setlength{\parindent}{0pt}
    \setlength{\parskip}{6pt plus 2pt minus 1pt}}
}{% if KOMA class
  \KOMAoptions{parskip=half}}
\makeatother
\usepackage{xcolor}
\setlength{\emergencystretch}{3em} % prevent overfull lines
\providecommand{\tightlist}{%
  \setlength{\itemsep}{0pt}\setlength{\parskip}{0pt}}
\setcounter{secnumdepth}{-\maxdimen} % remove section numbering
\ifLuaTeX
  \usepackage{selnolig}  % disable illegal ligatures
\fi
\IfFileExists{bookmark.sty}{\usepackage{bookmark}}{\usepackage{hyperref}}
\IfFileExists{xurl.sty}{\usepackage{xurl}}{} % add URL line breaks if available
\urlstyle{same}
\hypersetup{
  hidelinks,
  pdfcreator={LaTeX via pandoc}}

\author{}
\date{}

\begin{document}

\hypertarget{h.cdynd9tc2xhn}{%
\section{\texorpdfstring{{ARMv8 AArch64 Final Report: Group
5}}{ARMv8 AArch64 Final Report: Group 5}}\label{h.cdynd9tc2xhn}}

{}

\hypertarget{h.uu97cprfgtk5}{%
\subsection{\texorpdfstring{{Assembler}}{Assembler}}\label{h.uu97cprfgtk5}}

{}

{The assembler was split into several different files for handling
different stages of the assembly process: dealing with the types of
instructions and providing utility helper functions. In our program,
f}{irst we read the file and create an array of instructions (this is
where we also remove comments in our extension). Then we implement a
two-pass. On the first pass, we create an array of structs containing
labels and their addresses, then on the second pass we go through our
array of instructions deciding what to do based on whether the line is a
label, directive or instruction. If it is a label, we ignore it. If it
is a directive, we return the value in the directive. For instructions,
we use our array of structs and a linear search to replace the labels
with the appropriate address, then we use a function pointer to call the
appropriate function given a mnemonic. Each mnemonic has its own
function, however most of these call parsing functions which deal with
groups of similar instructions to avoid any code duplication. Each of
these parsing functions handles responsibilities such as determining
opcodes, setting flags, and calling the helper functions that determine
the values of registers and immediate values.}

{}

{Originally, we had issues writing to the file, as we were attempting to
write a binary string to the output file. We fixed this by writing the
full instruction in denary to the file. In addition, CLion had issues
with reading files which were read as null when we tried running the
program. After some research, we realised that we needed to use gcc
instead. Afterwards, we noticed that the vast majority of test cases
that failed were due to segmentation faults. In order to correct this,
we used the GDB debugger. We compiled the program using the -g switch
and then by using the run and backtrace commands, we could identify the
offending code. In conjunction with using numerous print statements, we
could pinpoint where the faults were happening. After our test cases
succeeded, some still turned out to be incorrect, so we used the diff
files and compared the hexadecimal values (which we converted into
binary) between the expected results and the actual results. By trying
to manually determine what the correct outputs should be, we could
understand why the expected results were what they were which then could
give us some insight into what was going wrong.}

{}

{There were a few interesting logic errors that we had to deal with. The
first was with the single data transfer instructions. Since ldr and str
had several different arguments for the various types of offsetting, the
parsing was quite complex. We initially created functions to handle
these various options which were largely working correctly, but the
parsing beforehand was often incorrect. This was because we had removed
all non-alphanumeric characters from the instructions, so that we did
not have any `\#' before immediate values or `:' after labels. But this
also removed the exclamation marks and square brackets we needed for
parsing, leading to us correcting our helper functions.}

{}

{Another interesting problem arose in a test case that had two labels
(wait1 and wait). Our initial approach to replacing labels was to find
all the substrings representing the label in the instruction string and
replacing the substrings with the appropriate address. As wait is a
substring of wait1, our assembler replaced wait1 with 0x41 since the
address of wait was 0x4, giving us an incorrect hexadecimal address. We
decided to change the way we were replacing labels by splitting the
instructions by commas and spaces into the mnemonic and arguments, then
explicitly comparing the arguments against the labels to check for
equality before making any replacements which fixed this error.}

{}

\hypertarget{h.3dri9a37vvl5}{%
\subsection{\texorpdfstring{{Raspberry
Pi}}{Raspberry Pi}}\label{h.3dri9a37vvl5}}

{}

{We structured our Raspberry Pi code as outlined in the specification.
We then used our assembler to assemble our code for the Raspberry Pi and
faced many challenges when debugging our code. We did not realise that
assembly arithmetic instructions cannot use immediate values as
arguments. We solved this by using `mov' instructions to load immediate
values into registers first. It was also not possible to access the
address of a label using the subset of the instruction set we had
implemented for the assembler. This meant we had to hardcode the address
of the request. We placed the request at the start of the file to ensure
the offset would remain the same even when we added more code. When we
}{ran}{~our code on the Raspberry Pi, the LED only }{blinked}{~once. We
used our emulator to test that the correct registers and memory
addresses were being written to. This surfaced more errors. When we were
trying to load the value of the mailbox status register, we actually
were only loading the address itself, not the value at the address.
Also, we used x1 to reference registers which meant that 64 bits were
loaded into memory, not 32 bits, so when using labels in code, we were
actually loading two lines into memory instead of one. Therefore we
switched to using w1 instead. When we tried our code on the LED again,
it flashed once and then flashed faster until it faded out. We tried to
isolate the cause by just loading our code to turn the LED on first, but
this failed too. Unfortunately at this point we didn't have time to
continue debugging and had to move on to the extension.}

{}

{Although we did not finish this part of the project, the process of
debugging helped us understand our assembly instructions in greater
depth. We also learnt how a mailbox system works when interfacing with
an LED or other Raspberry Pi input or output device. It was exciting to
have our first hardware-focused project, and the task gave us an
appreciation of how we could use our assembler and emulator together in
a real-life scenario. We could have approached this problem better to
give us more time to ensure the LED was flashing. For example, we could
have saved time debugging our assembly by ensuring we understood all the
assembly instructions before writing the code, rather than assuming what
each instruction does. Unfortunately, one aspect that slowed our
development speed was that we had to wait for our assembler to be
working before we could test our code. This left us very little time to
debug before the deadline.}

{}

\hypertarget{h.lnn9uewmn4ng}{%
\subsection{\texorpdfstring{{Extension 1:
Comments}}{Extension 1: Comments}}\label{h.lnn9uewmn4ng}}

{}

{For the first part of our extension, we added additional functionality
to the assembler that allows comments to be parsed. This worked by
removing the comments before the second pass so that they were not
translated into assembly code. We did this through two functions, one to
remove multi-line comments and one to remove single line comments. We
decided to parse the comments while reading in the file so that whenever
a line is read, any single line comments can be removed immediately. We
then remove any multi-line comments once the file is in the string
array.}

{}

{Parsing multi-line introduced some unexpected challenges. We had to
account for what happens when a multi-line comment starts or ends
halfway through a line, and when a multi-line comment ends on one line
and a new multi-line comment starts on the same line. We initially
tested our extension using simple test files for parsing single and
multiline comments. We tested both our functions in a separate project
so that we could more easily isolate errors. Finally, we tested our
extension on the Raspberry Pi led\_blink.s file which had multiple
comments to explain the flow of logic. We could have implemented more
test files to ensure our extension was working correctly and we could
have returned more personalised error messages highlighting unterminated
multi-line comments.}

\hypertarget{h.cqqy0hf95mzj}{%
\subsection{\texorpdfstring{{Extension 2:
Snake}}{Extension 2: Snake}}\label{h.cqqy0hf95mzj}}

{}

{We also built a second extension that allows you to play a snake game
in the console. The game is perfect to relax in the evening and pass
time when off WiFi. The goal of the game is to use the arrows keys to
move the snake to eat as many randomly generated apples as possible
without dying by crashing into walls or the snake's body itself. The
state of the game is represented using a 2D array of characters which
represents the area that the snake can explore. The snake is represented
using a LinkedList data structure where each node of the snake stores
its position on the grid and the next node of the snake. This was
essential to ensure the tail of the snake could be updated after each
loop.}

{}

{The snake changes direction by detecting keyboard input from the arrow
keys. We could not use the library function `scanf' to do this as it
waits until the enter key is pressed to return. Because snake is a
fast-paced game, we needed the program to respond immediately to a
keyboard input. Therefore, we used `getch' from the
\textless conio.h\textgreater{} library which keeps a buffer of keys
inputted and immediately returns the latest key press. Unfortunately
this function only works on Windows and this is a limitation we would
address given more time. We experimented with other libraries such as
\textless curses.h\textgreater{} and \textless ncurses.h\textgreater,
but came across the same issue.}

{}

{The snake `moves' by starting at the current head and creating a new
head of the linked list in the direction of the most recent key press.
If the new head lands on the apple, the tail of the snake is not freed
from memory which increases the length of the snake by one. Otherwise,
the tail is freed and the snake moves forwards by one unit. After each
loop, we clear the console and reprint the game board. The usual
commands to clear the console - }{clrsrc}{(), system(``}{cls''}{) and
system(``clear'') - did not work in CLion, so we had to use an escape
code instead. This solution works, but is not perfect as the cursor can
be seen moving around the screen as each frame is printed.}

{}

{Originally, the snake moved at the same speed by sleeping for a
constant duration after each loop. However, we decided to implement a
feature to increase the difficulty of the game for better players. The
speed of the snake now increases by a factor of 0.98 after each apple is
collected. We also calculate the player's skill level using how many
moves they made and the minimum number of moves that the player would
have to make to collect all the apples. The user can also compete
against themselves and their friends by playing multiple times and
keeping track of their high score.}

{}

{We have tested our game over 50 times and asked people in other teams
to find errors. For example, one tester discovered you could use the
arrow keys to immediately turn back on yourself and die. When running
the code, we also noticed that the apples were consistently appearing in
the same locations, despite using the random function. To fix this we
needed to set the seed of the generator to the unique time the code is
run. However, the system `rand' function has very limited randomness
which is another limitation we would address in further development. We
could have also tested our solution more thoroughly on operating systems
other than Windows and we could have implemented test functions to test
our helper functions.}

{}

\hypertarget{h.7uwhlvjrxtxq}{%
\subsection{\texorpdfstring{{Group
Reflection}}{Group Reflection}}\label{h.7uwhlvjrxtxq}}

{We split the team into two pairs with Virginia and Jack initially
working on the emulator whilst Bella and Aarit worked on the assembler.
The emulator was the first of these two tasks to be completed, so the
pair then started work on the Raspberry Pi and the extension. While
coding the Raspberry Pi, Virginia and Jack were able to communicate with
Bella and Aarit about how the assembler parsed the assembly language.
After the assembler was completed, Bella and Aarit were able to create a
second extension.}

{In order to coordinate all our work and ensure everyone is being held
accountable, we have all come into the university computer labs at least
4 days a week from 10am-6pm. We have all been sitting together and
maintained constant discussion and communication, allowing for effective
pair programming. This was advantageous, as it reduced coding errors and
we could learn from each other about C. Although it felt slower at the
time, we implemented functions faster than if we had coded individually
and then analysed each other's code in our spare time.}

{}

{After reading the specification, we noticed that, if we had four people
all working on the same part, there would be a lot of redundancy. One
person would often have to wait for another to finish their tasks before
they start their own. Instead, we decided to split the group into two
pairs with one working on the emulator and the other working on the
assembler. Since understanding of how the emulator worked was a
prerequisite to building the assembler, we - as a group - decided to
first discuss a broad approach to implementing the emulator.}

{There was a lot we learnt about how we could have improved. For
example, we started implementing the assembler too quickly without
planning and mapping the broader structure of the assembler. In
addition, we had not read the entirety of the specification before we
started and were instead trying to understand different sections in
isolation as we were going along. This made initial progress much
quicker, but debugging was harder as a result. It would have also been
useful to consider things such as what data types we should use at the
beginning. In hindsight, we could have used a program to manage which
tasks were left to do and we could have used git issues to highlight
bits of code to work on.}

\hypertarget{h.ttl365hle8uq}{%
\subsection{\texorpdfstring{{Individual
Reflections}}{Individual Reflections}}\label{h.ttl365hle8uq}}

\hypertarget{h.ismxnmdep3kw}{%
\subsubsection{\texorpdfstring{{Aarit:}}{Aarit:}}\label{h.ismxnmdep3kw}}

{Due to the way we had split the work between us, my understanding of
the spec was quite useful in the project as I was able to read ahead and
understand how various parts worked whilst Bella was implementing what
we discussed. This allowed me to then fully understand what to do next
when Bella was ready to implement the next part. This was also useful
for the Raspberry Pi when other members of the group who were less
familiar with assembly code were able to ask me about whether certain
instructions were syntactically correct. However, I don't have much
experience in programming and most of that experience has been with
languages such as Python and Swift where the rules are far less strict
and the compilers/interpreters do a lot more work. I've never dealt with
concepts such as memory allocation, so when debugging and writing my
helper functions, I frequently ran into segmentation faults which I
found very difficult to fix. I occasionally also had issues dealing with
git and running the test suite due to my lack of prior experience as
well. Next time, I would make a more active effort at the beginning of
the project to learn about C in my free time. In addition, I would try
to become more comfortable with using git and the terminal which I did
not do at all until I had to start running tests and debugging.}

\hypertarget{h.9tz4i4mcmyx4}{%
\subsubsection{\texorpdfstring{{Jack:}}{Jack:}}\label{h.9tz4i4mcmyx4}}

{I had no experience with C or a Raspberry Pi before this project unlike
some members of the group, so was initially anxious that I would have
lots to learn very quickly. However, because we employed pair
programming, I learnt lots from my peers at the start of the project and
quickly got up to speed. I had also not used Git before in a team, but
was surprised how easy it was to work together and merge conflicts.
Initially, I underestimated by many days the time it would take to debug
our emulator code. Although C was at times frustrating to debug, it was
very rewarding when our code worked. The group feedback highlighted that
one of my strengths was helping the group debug code. I made sure we
didn't spend too much time trying to debug a single test case which
helped us stay motivated. I also ensured I knew what every line of our
code does and periodically removed redundant lines to keep our code
readable. However this approach caused errors down the line where I had
removed break statements from a switch not knowing they were necessary.
This would also be an unsustainable approach on larger projects and I
need to learn to trust other group members to maintain their code.
Another weakness was that I should have taken breaks from coding more
often than I did, which would have increased my productivity. Next time,
I would set myself a smaller problem to ensure I knew the programming
language before the project started to reduce such trivial errors. We
should have also implemented testing functions for our code when writing
it instead of when debugging. This would have saved time searching for
errors in our source code and will be very important for future bigger
projects.}

\hypertarget{h.vo7j1qawt1os}{%
\subsubsection{\texorpdfstring{{Virginia:}}{Virginia:}}\label{h.vo7j1qawt1os}}

{Before starting this project, I had not previously coded in C or any
very similar languages - there were a lot of new features, like memory
allocation, that I did not know how to implement. Due to this, I spent a
little time before starting the project to learn the basic commands and
tried a few small coding challenges. Despite giving me a little help at
the start, I do not feel they truly aided me, possibly due to the scale
of this project; however, this has meant as the project has gone on, my
skills have been much developed. Through my work on the emulator and the
snake extension, I have found I have good analytical skills - splitting
up any functions into smaller and smaller chunks that are more
manageable, and despite finding debugging frustrating, gotten better at
recognising potential flaws in my code and systematically testing to
find them. I found that I have a tendency to overcomplicate some
functions, but have also found that discussing them with the others in
my project tends to mitigate this. Next time, I would maintain our
communication level, as this was particularly useful in terms of
debugging, and I would try to improve my organisation skills, in order
to be more efficient - i.e. not spending too long on one small section.}

\hypertarget{h.nvh7vvlf6lpp}{%
\subsubsection{\texorpdfstring{{Bella:
}}{Bella: }}\label{h.nvh7vvlf6lpp}}

{Before starting the project I only had a basic understanding of C
having coded in it briefly two years previously, although I had worked
with Git quite extensively before. I felt that the pair programming
approach worked incredibly well as it allowed me to have a chance to
improve my coding by gaining more experience and taking on the challenge
of programming difficult functions. The other members of the team
provided a lot of support throughout the project and I have learnt a
huge amount in terms of syntax and also in terms of analytical thinking
and approaches to coding. As a result, I have found that not only have
my C programming skills improved greatly but also my general approach
towards programming. I found when testing our work that I had errors
sometimes in casting between data types which was something that i
previously had not considered, having worked in languages such as Java,
C\# and Python, all of which are nice in terms of casting. This forced
me to understand how casting actually worked so that I could write
corresponding helper functions which performed the correct casting. I
underestimated the size of the challenge that would be debugging the
code, having misjudged how difficult segmentation faults would be to fix
in the beginning before we created a better system to locate and fix
them. I also relied too heavily on the IDE at first and when an issue
occurred with IDE access to files that meant that we were unable to
execute our code in the IDE and instead had to do so in the terminal - I
had to improve my ability to use the terminal very quickly. Next time, I
would continue our current communication and split as I felt that it was
incredibly effective throughout the project. ~I would also continue a
similar work day to the one that we established within this project as I
felt that our 10-4 work day, 4 days a week, was very effective and
increased the whole group\textquotesingle s productivity. However, I
would also test my code with unit tests as I am programming rather than
leaving it all to the end which ultimately I feel wasted time and made
the task of debugging a lot harder.}

\end{document}

